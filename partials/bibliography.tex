\chapter*{Bibliographie}
\setstretch{1.42}

\section*{Encyclopédies}
\section*{Ouvrages}
Y. Algan, P. Cahuc. \emph{La société de défiance: comment le modèle social français s’autodétruit}. Collection du CEPREMAP. Rue d’Ulm, :105, 2007.\par{}

\section*{Articles de revues et doctrine}

L.-M. Augagneur. Le numérique, outil de simplification de la pratique du droit des affaires. \emph{Droit et patrimoine}, (323):1-6, 2022.\par{}

T. Barton, G. Berger-Walliser, and H. Haapio. Contracting for innovation and innovating contracts: an overview and introduction to the special issue. \emph{Journal of Strategic Contracting and Negotiation}, 2:3–9, 2016.

L. Benezech. L’exigence d’accessibilité et d’intelligibilité de la loi : retour sur vingt ans d’existence. \emph{Rev. fr. droit const.}, 1283(3):541, 2020.\par{}

G. Brenas. L’intérêt du legal design pour les professionnels du droit. \emph{Rev. prat. de la prospective et de l’innovation}, (2), 2019.\par{}

L. Cadiet. Les marchés du droit. Conclusions. \emph{Rev. int. droit écon.}, (4):127, 2017.\par{}

F. Creux-Thomas. Legal design, gadget or opportunity for lawyers ? \emph{JCP E.}, (51), 2019.\par{}

B. Dondero. Legal design. parler de design à propos du droit a-t-il un sens ? \emph{JCP G.}, (4), 2019.\par{}

L. Drai, M. Béjean. Innovation, collaboration et droit. \emph{Rev. fr. gest.}, 43(269): 185–206, 2017.\par{}

M. Hastak, A. Levy. Consumer comprehension of financial privacy notices : A report on theresults of thequantitative testing. \emph{Interagency Notice Project}, 2008.\par{}

D. Irimia. Pour une nouvelle branche de droit ? La traduction juridique, du droit au langage , \emph{Éla. Études de linguistique appliquée}, 183(3):329-341, 2016.\par{}

L. Krawietz, J. Edelman, J. Santuber, B. Owoyele. The need for a Legal Design metatheory for the emergence of change in the creative legal society. \emph{Legal Design as Academic Discipline: Foundations, Methodology, Applications}, :16, 2018.\par{}

S. Lapisardi. Qu’est-ce que le langage juridique clair. \emph{JCP E.}, (51), 2019.\par{}

J. Onimus. Puissances de l’image. \emph{Étrangeté de l’art}, :89–106, 1992.

E. Pitkäsalo, L. Kalliomaa-Puha. Democratizing access to justice: the comic contract as intersemiotic translation. \emph{Translation Matters}, 1:30–42, 2019.\par{}

I. Pollach. What’s wrong with online privacy policies ? \emph{Commun. ACM}, 50(9): 103–108, 2007.

M. Potel-Saville. Pour en finir avec les préconceptions sur le langage juridique clair. \emph{Rev. prat. de la prospective et de l'innovation}, (2):1-13, 2020. 

L. Shepherd, K. Renaud. How to make privacy policies both GDPR-compliant and usable. \emph{2018 International Conference On Cyber Situational Awareness, Data Analytics And Assessment (Cyber SA)}, :1-8, 2018.\par{}

L. Zunino. Crise et accélération de la transformation digitale du secteur juridique. \emph{Droit et patrimoine}, (309):1-5, 2021.\par{}

\section*{Thèses universitaires}

\section*{Codes}
Code de procédure civile\par{}

\section*{Règlements et directives européennes}
Règlement (UE) 2016/679 relatif à la protection des personnes physiques à l’égard du traitement des données à caractère personnel et à la libre circulation de ces données, et abrogeant la directive 95/46/CE. Parlement européen et Conseil. 27 avril 2016.\par{}

\section*{Chartes et recommandations}

Agence Française Anti-corruption. \emph{Recommandations destinées à aider les personnes morales de droit public et de droit privé à prévenir et à détecter les faits de corruption, de trafic d'influence, de concussion, de prise illégale d'intérêt, de détournement de fonds publics et de favoritisme}. :10, 2017.\par{}

Direction de l'information légale et administrative. \emph{Charte orthotypographique du Journal officiel}. Journal officiel de janvier 2021,  :27, 2021.\par{}

\section*{Jurisprudences}
Cour d'appel, Paris, pôle 5, chambre 4, 30 Juin 2021 – \no 20/18373\par{}

\section*{Rapports}
Hodge Jones \& Allen. Unjust kingdom : UK perceptions of the legal and justice system. \emph{Innovation in law report 2015}, 2015.\par{}

\section*{Documents non publiés}

L. Montant. Legal design : vers une transformation radicale de l’expérience client des cabinets d’avocats. \emph{Dalloz Actualités}, 2020.\par{}

R. Yankovskiy. Legal design: New thinking and new challenges. :76, 2019.

\section*{Presse généraliste et documents de travail}

M. Hagan. Law by design. URL https://lawbydesign.co.\par{}

\newpage